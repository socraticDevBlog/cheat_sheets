\documentclass[11pt, landscape]{article}
\usepackage{multicol}
\usepackage{calc}
\usepackage{ifthen}
\usepackage[landscape]{geometry}
\usepackage{amsmath,amsthm,amsfonts,amssymb}
\usepackage{color,graphicx,overpic}
\usepackage{hyperref}

%pour le francais : accents et tout
%\usepackage[french]{babel}
%\usepackage[utf8]{inputenc}

%nous permet d'utiliser la fonction de commentaires
\usepackage{verbatim}


\pdfinfo{
	/Title (Python cheat sheet)
	/Creator (TeX)
	/Producer (pdfTeX 1.40.0)
	/Author (SocraticDev)
	/Subject (Example)
	/Keywords (python)}

% This sets page margins to .5 inch if using letter paper, and to 1cm
% if using A4 paper. (This probably isn't strictly necessary.)
% If using another size paper, use default 1cm margins.
\ifthenelse{\lengthtest { \paperwidth = 11in}}
{ \geometry{top=.5in,left=.5in,right=.5in,bottom=.5in} }
{\ifthenelse{ \lengthtest{ \paperwidth = 297mm}}
	{\geometry{top=1cm,left=1cm,right=1cm,bottom=1cm} }
	{\geometry{top=1cm,left=1cm,right=1cm,bottom=1cm} }
}

% Turn off header and footer
\pagestyle{empty}

% Redefine section commands to use less space
\makeatletter
\renewcommand{\section}{\@startsection{section}{1}{0mm}%
	{-1ex plus -.5ex minus -.2ex}%
	{0.5ex plus .2ex}%x
	{\normalfont\large\bfseries}}
\renewcommand{\subsection}{\@startsection{subsection}{2}{0mm}%
	{-1explus -.5ex minus -.2ex}%
	{0.5ex plus .2ex}%
	{\normalfont\normalsize\bfseries}}
\renewcommand{\subsubsection}{\@startsection{subsubsection}{3}{0mm}%
	{-1ex plus -.5ex minus -.2ex}%
	{1ex plus .2ex}%
	{\normalfont\small\bfseries}}
\makeatother

% Don't print section numbers
\setcounter{secnumdepth}{0}

\setlength{\parindent}{0pt}
\setlength{\parskip}{0pt plus 0.5ex}

%My Environments
\newtheorem{example}[section]{Example}

% for code snippets
\usepackage{listings}
\usepackage{color}

\definecolor{dkgreen}{rgb}{0,0.6,0}
\definecolor{gray}{rgb}{0.5,0.5,0.5}
\definecolor{mauve}{rgb}{0.58,0,0.82}

\lstset{frame=tb,
  language=Java,
  aboveskip=3mm,
  belowskip=3mm,
  showstringspaces=false,
  columns=flexible,
  basicstyle={\small\ttfamily},
  numbers=none,
  numberstyle=\tiny\color{gray},
  keywordstyle=\color{blue},
  commentstyle=\color{dkgreen},
  stringstyle=\color{mauve},
  breaklines=true,
  breakatwhitespace=true,
  tabsize=3
}
% -----------------------------------------------------------------------

\begin{document}
	\raggedright
	\footnotesize
	\begin{multicols}{3}
		
		% multicol parameters
		% These lengths are set only within the two main columns
		%\setlength{\columnseprule}{0.25pt}
		\setlength{\premulticols}{1pt}
		\setlength{\postmulticols}{1pt}
		\setlength{\multicolsep}{1pt}
		\setlength{\columnsep}{2pt}
		
		\begin{center}
			\Large{\underline{Python CheatSheet}} \\
		\end{center}
		
		\section {Usual dependencies}
                     \subsection {Virtual environnements}
			\begin {itemize}
				\item built-in feature of the language since version 3.4
				\item {create a new virtual environment}
                               		\begin {verbatim}
					python -m venv myVenvName
				\end {verbatim}
				\item {entering into the virtual environnement}
				\item {linux}
                               		\begin {verbatim}
					source ./myVenvName/bin/activate
				\end {verbatim}
				\item {Windows when ran in Powershell}
                               		\begin {verbatim}
					./myVenvName/Scripts/activate.ps1
				\end {verbatim}
				\item {exit the virtual environnement}
                                           	\begin {verbatim}
						deactivate
					\end {verbatim}
			\end {itemize}
			
		\subsection{Installing PIP (python package manager)}

		\begin{enumerate}
			\item Download the boostrap installer file \begin {verbatim}get-pip.py\end {verbatim}
                                \item put it near your Python executables
                                \item execute the file with Python
                               		\begin {verbatim}
					python get-pip.py 
				\end {verbatim}
			 
		\end{enumerate}

		\section {Utils}
		\begin {itemize}
			\item {locating actual Python interpreter}

				\begin{lstlisting}
>>> import os
>>> import sys
>>> os.path.dirname(sys.executable)
'C:\\Python25'
				\end{lstlisting}
		\end {itemize}
		
		% You can even have references
		\rule{0.3\linewidth}{0.25pt}
		\scriptsize
		\bibliographystyle{abstract}
		\bibliography{refFile} \\
	           last compilation : \today
	\end{multicols}
\end{document}