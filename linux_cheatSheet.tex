\documentclass[11pt, landscape]{article}
\usepackage{multicol}
\usepackage{calc}
\usepackage{ifthen}
\usepackage[landscape]{geometry}
\usepackage{amsmath,amsthm,amsfonts,amssymb}
\usepackage{color,graphicx,overpic}
\usepackage{hyperref}
\usepackage[T1]{fontenc}

%pour le francais : accents et tout
%\usepackage[french]{babel}
%\usepackage[utf8]{inputenc}

%nous permet d'utiliser la fonction de commentaires
\usepackage{verbatim}

\pdfinfo{
	/Title (Linux bash cheatSheet)
	/Creator (TeX)
	/Producer (pdfTeX 2.9.6499)
	/Author (SocraticDev)
	/Subject (Linux)
	/Keywords (linux, bash, commands, basics)}

% This sets page margins to .5 inch if using letter paper, and to 1cm
% if using A4 paper. (This probably isn't strictly necessary.)
% If using another size paper, use default 1cm margins.
\ifthenelse{\lengthtest { \paperwidth = 11in}}
{ \geometry{top=.5in,left=.5in,right=.5in,bottom=.5in} }
{\ifthenelse{ \lengthtest{ \paperwidth = 297mm}}
	{\geometry{top=1cm,left=1cm,right=1cm,bottom=1cm} }
	{\geometry{top=1cm,left=1cm,right=1cm,bottom=1cm} }
}

% Turn off header and footer
\pagestyle{empty}

% Redefine section commands to use less space
\makeatletter
\renewcommand{\section}{\@startsection{section}{1}{0mm}%
	{-1ex plus -.5ex minus -.2ex}%
	{0.5ex plus .2ex}%x
	{\normalfont\large\bfseries}}
\renewcommand{\subsection}{\@startsection{subsection}{2}{0mm}%
	{-1explus -.5ex minus -.2ex}%
	{0.5ex plus .2ex}%
	{\normalfont\normalsize\bfseries}}
\renewcommand{\subsubsection}{\@startsection{subsubsection}{3}{0mm}%
	{-1ex plus -.5ex minus -.2ex}%
	{1ex plus .2ex}%
	{\normalfont\small\bfseries}}
\makeatother

% Don't print section numbers
\setcounter{secnumdepth}{0}

\setlength{\parindent}{0pt}
\setlength{\parskip}{0pt plus 0.5ex}

%My Environments
\newtheorem{example}[section]{Example}
% -----------------------------------------------------------------------

\begin{document}
	\raggedright
	\footnotesize
	\begin{multicols}{3}
		
		% multicol parameters
		% These lengths are set only within the two main columns
		%\setlength{\columnseprule}{0.25pt}
		\setlength{\premulticols}{1pt}
		\setlength{\postmulticols}{1pt}
		\setlength{\multicolsep}{1pt}
		\setlength{\columnsep}{2pt}
		
		\begin{center}
			\Large{\underline{linux CheatSheet}} \\
		\end{center}
	
		\section {Top-level Directories}
			\begin {itemize}
				\item / : root
				\subitem /etc : Program Configuration Files 
				\subitem /var : Frequently changing content (ex. logs)  
				\subitem /home : User account files  
				\subitem /sbin : System binary files  
				\subitem /bin : User binary files  			
				\subitem /lib : Shared libraries
				\subitem /usr : Third-party libraries	 			
			\end {itemize}
		
		\section{Basic Commands}
		
			\subsection {ls}
				\begin {itemize}
					\item lists content of a directory
						\subitem \texttt{ls [/aDirectoryName]}
					\item lists content with the long flag ${-l}$ and human-readble flag ${h}$ displays file permissions and size information
						\subitem \texttt{\$ ls -lh /users/max/documents}
					\item lists with the recursive flag ${-R}$ everything in and under a directory
						\subitem \texttt{\$ ls -R /users/max/documents}
				\end {itemize}
			
			\subsection {pwd (present working directory)}
				\begin {itemize}
					\item print name of current/working directory
					\item using flag Logical ${-L}$ to include symlinks
				\end {itemize}
			
			\subsection {cd (change directory)}
				 \begin {itemize}
				 	\item ex. : moving to root directory :
				 	\subitem \texttt{\$ cd \textbackslash}
				 \end {itemize}			
				 
			\subsection {rm (remove)}
				\begin {itemize}
					\item removing a file :
						\subitem \texttt{\$ rm myFile}
					\item removing an empty directory:
						\subitem \texttt{\$ rmdir myEmptyDirectory}
					\item removing a non-empty directory and it contents recursively :
						\subitem \texttt{rm -r myDirectory}
				\end {itemize}
				 
			\subsection {cp (copy)}
				\begin {itemize}
					\item copy a file to a specified directory
						\subitem \texttt{\$ cp myFile /home/max/anotherDirectory}
				\end {itemize}
				 
			\subsection {mv (move)}	
				\begin {itemize}
					\item move a file to a specified directory
						\subitem \texttt{\$ cp myFile /home/max/anotherDirectory}
				\end {itemize}				 
			
		\section {Other Commands}
			
			\subsection {cat (concatenate)}
				\begin {itemize}
					\item prints file content to output
				\end {itemize}
			
			\subsection {less}
				\begin {itemize}
					\item quickly display less than a complete file contents
					\item scroll document's content with arrows
					\item quit by typing 'q'
				\end {itemize}				
			
			\subsection {man}		
				\begin {itemize}
					\item brings up a one-page interface to reference manuals about a linux command
						\subitem ex. : \texttt{\$ man ls}
					\item type \textbackslash (backslash) followed by keywords to search into document
						\subitem move forward to next keyword by typing 'n'
					\item quit by typing 'q'
				\end {itemize}
				
			\subsection {touch}		
				\begin {itemize}
					\item updates file timestamps
					\item if file doesn't exist it will create it
				\end {itemize}
			
			\subsection {stat (status)}		
				\begin {itemize}
					\item displays file or file system status
						\subitem displays file's inode (metadata) information 
				\end {itemize}
				
			\subsection {| (pipe)}	
			\begin {itemize}
				\item creates a unidirectional data channel
				\item takes the output of a commande to feed it as the input of another
					\subitem \texttt{\$ journalctl | grep myFile.php}
				\item you can chain as many pipes as you wish
					\subitem \texttt{\$ journalctl | grep myFile.php | grep error}
			\end {itemize}	
			
			\subsection {info}		
				\begin {itemize}
					\item reading documentation in Info format
					\item useful, when you don't know the name of a command you want to use
					\item click Enter on underlined text to follow links
						\subitem 'u' will bring you back one level
						\subitem 'q' to exit 
				\end {itemize}
				
			\subsection {journalctl}		
				\begin {itemize}
					\item tool to query the contents of the systemd journal
						\subitem ideally do filter query with grep
							\subsubitem \texttt{\$ journalctl | grep myPage.php} 
				\end {itemize}
				
			\subsection {arch}		
				\begin {itemize}
					\item prints your system architecture (ex: x86\_64) 
				\end {itemize}
				
		\section {Keyboard Shortcuts}
		
			\subsection {TAB (completion)}
				\begin {itemize}
					\item press TAB to complete a command
						\subitem \texttt{\$ touch myNewFile}
						\subitem \texttt{\$ rm my<TAB>}
				\end {itemize}
				
		\section {Special Characters}
		
			\subsection {* (gobbling)}
				\begin {itemize}
					\item wildcard used to designate all files in a directory
						\subitem \texttt{\$ mv * /home/max/archive}
					\item wildcard used to designate any characters
						\subitem \texttt{\$ git add linux*.tex, linux*.pdf}
				\end {itemize}
				
			\subsection {? (question mark)}
				\begin {itemize}
					\item represents or matches a single occurrence of any character
						\subitem there are files named 'file1', 'file2', 'file3'
						\subitem \texttt{\$ rm file? /home/max/documents}
				\end {itemize}
				
			\subsection {\textbackslash (backslash)}
				\begin {itemize}
					\item The backslash character can be used to conveniently break a long command into multiple lines on the command line.
						\subitem \texttt{\# yum install lxc lxc-templates \textbackslash \\ libcaps-devel libcgroup busybox}	
				\end {itemize}				

		\section {System Administration} 
			\subsection {systemd}
				\begin {itemize}
					\item first process to run on a system
					\item show all services and processes running
						\subitem \texttt{\$ systemctl list-units --type service --state running}
						\subitem {or}
						\subitem \texttt{\$ systemctl --no-pager | grep service | grep running | column -t}
					\item show all installed unit-files 
						\subitem \texttt{\$ systemctl list-units --type service}
						\subitem {or}
						\subitem \texttt{\$ systemctl list-units --type service --state running --no-legend}
					\item show processes that executed then exited
						\subitem \texttt{\$ systemctl list-units --type service --state exited}
					\item show processes that have failed
						\subitem \texttt{\$ systemctl list-units --type service --state failed}
					\item *In order to pass the output to $stdout$ instead of a pager
						\subitem \texttt{add argument `---no-pager'}
				\end {itemize}

			
		% You can even have references
		\rule{0.3\linewidth}{0.25pt}
		\scriptsize
		\bibliographystyle{abstract}
		\bibliography{refFile}\\
		Last compiled : (\today)
	\end{multicols}
\end{document}